\chapter*{Conclusione}
\addcontentsline{toc}{chapter}{Conclusione} % add the chapter to the index
È ragionevole assumere che la decentralizzazione della rete elettrica sia un processo inevitabile che coinvolgerà un numero sempre crescente di utenze, 
sia nel ruolo di consumatori che nel ruolo di produttori. \\

Fra le possibili soluzioni, quella di usare la blockchain come infrastruttura in grado di supportare questo mercato sempre più decentralizzato sembra essere particolarmente promettente. \\
Ad una analisi approfondita \gls{ew} risulta essere un progetto con una visione ed un focus chiari, guidato da persone che hanno le conoscenze necessarie per creare una piattaforma ad hoc per questo settore. \\
Proprio in virtù di ciò, le tecnologie che \gls{ew} impiega non sono necessariamente innovative.
Si cerca, al contrario, di seguire ed implementare quanti più standard possibile, e partire da questi per fornire dei servizi che rendano appetibile l'intera infrastruttura. \\
La scelta di rendere tutto il materiale, le implementazioni e le librerie open-source è in linea con la volontà di fornire degli strumenti che facilitino la vita agli sviluppatori, 
sui quali poi ricade il compito di creare \gls{dapp} e servizi per l'utente finale. \\
È bene, infine, tenere a mente che \gls{ew}, al momento della stesura di questo documento, è un progetto in divenire, e, per questo, soggetto a continui cambiamenti.
Nonostante ciò, penso che \gls{ew} fornisca quantomeno un ottimo caso di studio che valga la pena approfondire per chi fosse interessato ad affrontare la problematiche trattate nell'introduzione del documento.
