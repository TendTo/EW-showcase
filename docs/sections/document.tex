
% People and organizations have the ability to trade directly with each other, without banks or any other brokerage organization in the middle, 
% as is the case with conventional transactions. Because of this, the blockchain is expected to change the way we perform global value transactions. 
% Therefore, it is important to explore what this new technology may offer in the energy sector. 
% It is apparent that in this work we will focus on energy transactions and power grid operation in general, instead of pure monetary transactions.

% We should point out that although we almost exclusively consider the power grid, 
% we believe that the facts presented and the lessons learned as well as the concluding remarks apply to other energy sectors as well, with no particular difficulty.

% —The concept of smart grid has been introduced
% as a new vision of the conventional power grid to figure out
% an efficient way of integrating green and renewable energy
% technologies. In this way, Internet-connected smart grid, also
% called energy Internet, is also emerging as an innovative approach
% to ensure the energy from anywhere at any time. The ultimate
% goal of these developments is to build a sustainable society.
% However, integrating and coordinating a large number of growing
% connections can be a challenging issue for the traditional centralized grid system. Consequently, the smart grid is undergoing a
% transformation to the decentralized topology from its centralized
% form. On the other hand, blockchain has some excellent features
% which make it a promising application for smart grid paradigm.
% In this paper, we aim to provide a comprehensive survey on
% application of blockchain in smart grid. As such, we identify
% the significant security challenges of smart grid scenarios that
% can be addressed by blockchain. Then, we present a number of
% blockchain-based recent research works presented in different
% literatures addressing security issues in the area of smart grid.
% We also summarize several related practical projects, trials, and
% products that have been emerged recently. Finally, we discuss
% essential research challenges and future directions of applying
% blockchain to smart grid security issues.



% \begin{figure}[t] %[t] per inserire la figura in cima alla pagina
% 	\includegraphics[width=1\linewidth]{university_logo.png} %width=1\linewidth per scalare l'immagine alle dimensioni opportune. E' possibile ridurre le dimensioni dell'immagine inserendo un numero minore di 1
% 	\caption{Questo è un esempio di didascalia.}
% 	\label{fig:immagine} %per poter richiamare l'immagine nel testo
% \end{figure}


% \begin{lstlisting}[caption={Didascalia},label=lst:multicorrelation]
% double multicorrelation_ncc(template_obj,candidate_obj) {
% double similarity = ncc(template_obj,candidate_obj);
% if (similarity<t) {
% split template_obj into blocks: template_obj[9];
% split candidate_obj into blocks:  candidate_obj[9];
% similarity=0;
% for (i=0; i<9; i++)
% similarity+=ncc(template_obj[i],candidate_obj[i]);
% similarity/=9;
% }
% else
% return similarity;
% }
% \end{lstlisting}

% \begin{table}[ht]
% 	\begin{center}
% 		\caption{Tabella Generica}\label{tab:gen}
% 		\begin{tabular}{lcccccc}

% 			& & & & & & \\ \hline
% 			\multicolumn{1}{|c}{}  & \multicolumn{5}{|c|}{\bf Y} &
% 			\multicolumn{1}{c|}{} \\

% 			\multicolumn{1}{|c}{\bf X} & \multicolumn{1}{|c}{$1$} & $\ldots$ & $j$ & $\ldots$ &
% 			\multicolumn{1}{c|}{$k$} & \multicolumn{1}{c|}{$\Sigma_{j=1}^k{X_{ij}}$} \\ \hline

% 			\multicolumn{1}{|c}{$1$} & \multicolumn{1}{|c}{$X_{11}$} & $\ldots$ & $X_{1j}$ &
% 			$\ldots$ & \multicolumn{1}{c|}{$X_{1k}$} &
% 			\multicolumn{1}{c|}{$m$}\\

% 			\multicolumn{1}{|c}{$\vdots$} & \multicolumn{1}{|c}{$\vdots$} &  & $\vdots$ &  &
% 			\multicolumn{1}{c|}{$\vdots$} &
% 			\multicolumn{1}{c|}{$\vdots$}\\

% 			\multicolumn{1}{|c}{$i$} & \multicolumn{1}{|c}{$X_{i1}$} & $\dots$ & $X_{ij}$ &
% 			$\ldots$ & \multicolumn{1}{c|}{$X_{ik}$} &
% 			\multicolumn{1}{c|}{$m$} \\

% 			\multicolumn{1}{|c}{$\vdots$} & \multicolumn{1}{|c}{$\vdots$} &
% 			& $\vdots$ &  & \multicolumn{1}{c|}{$\vdots$} & \multicolumn{1}{c|}{\vdots} \\

% 			\multicolumn{1}{|c}{$n$} & \multicolumn{1}{|c}{$X_{n1}$} & $\dots$ & $X_{nj}$ &
% 			$\ldots$ & \multicolumn{1}{c|}{$X_{nk}$} & \multicolumn{1}{c|}{$m$}\\ \hline

% 		\end{tabular}
% 	\end{center}
% \end{table}