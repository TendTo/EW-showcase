\newglossaryentry{did}
{
    name={Decentralized IDentifiers (DID)},
    first={Decentralized IDentifiers (DID)},
    text={DID},
    plural={DIDs},
    description={ 
        Identificatori che permettono di ottenere un'identità digitale che sia verificabile e decentralizzata. 
        Si basano sullo standard stabilito da W3C. 
        Nella pratica, si tratta di URIs che puntano ad un DID document. 
    }
}
\newglossaryentry{der}
{
    name={Distributed Energy Resource (DER)},
    first={Distributed Energy Resource (DER)},
    text={DER},
    plural={DERs},
    description={ 
        Risorse fisiche o virtuali in grado di offrire un contributo attivo alla rete elettrica. 
        Sono esempi di DER impianti fotovoltaici, turbine eoliche, macchine elettriche, batterie, etc. 
    }
}
\newglossaryentry{dsm}
{
    name={Demand Side Management (DSM)},
    first={Demand Side Management (DSM)},
    text={DSM},
    plural={DSMs},
    description={ 
        Un insieme di azioni volte a gestire in maniera efficiente i consumi di un sito, 
        al fine di ridurre i costi sostenuti per l’approvvigionamento di energia elettrica, 
        per gli oneri di rete e per gli oneri generali di sistema, incluse le componenti fiscali. 
    }
}
\newglossaryentry{irec}
{
    name={International Renewable Energy Certificate (I-REC)},
    first={International Renewable Energy Certificate (I-REC)},
    text={I-REC},
    plural={I-RECs},
    description={ 
        Standard usato a livello internazionale (in oltre 45 stati) per certificare l'energia proveniente da fonti rinnovabili. 
    }
}
\newglossaryentry{poa}
{
    name={Proof of Authority (PoA)},
    first={Proof of Authority (PoA)},
    text={PoA},
    plural={PoAs},
    description={ 
        Algoritmo di consenso basato su una cerchia ristretta di nodi fidati, i validatori, 
        che sono gli unici in grado di aggiungere nuovi blocchi alla blockchain. 
    }
}
\newglossaryentry{pos}
{
    name={Proof of Stake (PoS)},
    first={Proof of Stake (PoS)},
    text={PoS},
    plural={PoSs},
    description={ 
        Algoritmo di consenso che utilizza un processo di elezione pseudo-casuale per selezionare un nodo che agirà da validatore del blocco successivo, 
        in base a una combinazione di fattori che possono includere periodo di staking, randomizzazione e fondi di proprietà del nodo. 
    }
}
\newglossaryentry{evm}
{
    name={Ethereum Virtual Machine (EVM)},
    first={Ethereum Virtual Machine (EVM)},
    text={EVM},
    plural={EVMs},
    description={ 
        Macchina virtuale in grado di modificare lo stato della blockchain a cui appartiene. 
        Permette l'esecuzione degli smart contract. 
    }
}
\newglossaryentry{erc}
{
    name={Ethereum Request for Comment (ERC)},
    first={Ethereum Request for Comment (ERC)},
    text={ERC},
    plural={ERCs},
    description={ 
        Proposte volte a migliorare Ethereum, generalmente aggiungendo nuove funzionalità e standard. 
    }
}
\newglossaryentry{ewns}
{
    name={Energy Web Name Service (EWNS)},
    first={Energy Web Name Service (EWNS)},
    text={EWNS},
    plural={EWNSs},
    description={ 
        Servizio simile ad un DNS presente su \gls{ewc} che associa ad un indirizzo esadecimale della blockchain un nome di dominio. 
    }
}
\newglossaryentry{ewc}
{
    name={Energy Web Chain (EWC)},
    first={Energy Web Chain (EWC)},
    text={EWC},
    plural={EWCs},
    description={ 
        Blockchain sviluppata da Energy Web. Si tratta di un fork di Ethereum 
    }
}
\newglossaryentry{eac}
{
    name={Energy Attribute Certificate (EAC)},
    first={Energy Attribute Certificate (EAC)},
    text={EAC},
    plural={EACs},
    description={ 
        Certificati emessi come prova di elettricità prodotta da fonti rinnovabili. 
        Ogni EAC certifica che 1MWh (o talvolta un KWh) sia stato generato e immesso nella rete da una fonte rinnovabile. 
        Alcuni degli standard più comuni sono il Guarantees of Origin (EU), l'I-REC (global) e il REC (US/Canada). \\
        Alcuni termini legati agli EAC sono:
        \begin{itemize}
            \item \textbf{Redeemed, Claimed o Cancelled}: EAC che è stato assegnato o distrutto e che non può essere rivenduto
            \item \textbf{Bundled Certificates}: contratto che vende sia energia che il certificato ad essa associato
            \item \textbf{Unbundled (...)}: contratto che vende solo energia o un EAC, ma non entrambi
        \end{itemize}
    }
}
\newglossaryentry{kms}
{
    name={Key Management System (KMS)},
    first={Key Management System (KMS)},
    text={KMS},
    plural={KMSs},
    description={ 
        Sistema per la gestione delle chiavi. 
    }
}
\newglossaryentry{spf}
{
    name={Single Point of Failure (SPO)},
    first={Single Point of Failure (SPO)},
    text={SPO},
    plural={SPOs},
    description={ 
        Punto debole di un architettura il cui malfunzionamento può causare anomalie o addirittura la cessazione del servizio da parte del sistema. 
    }
}
\newglossaryentry{ipfs}
{
    name={InterPlanetary File System (IPFS)},
    first={InterPlanetary File System (IPFS)},
    text={IPFS},
    plural={IPFSs},
    description={ 
        Protocollo di comunicazione e una rete peer-to-peer per l'archiviazione e la condivisione di dati in un file system distribuito. 
    }
}
\newglossaryentry{dht}
{
    name={Distributed Hash Tables (DHT)},
    first={Distributed Hash Tables (DHT)},
    text={DHT},
    plural={DHTs},
    description={ 
        Classe di sistemi distribuiti decentralizzati che partizionano l'appartenenza di un set di chiavi tra i nodi partecipanti, 
        e possono inoltrare in maniera efficiente i messaggi all'unico proprietario di una determinata chiave. 
    }
}
\newglossaryentry{sla}
{
    name={Service-Level Agreement (SLA)},
    first={Service-Level Agreement (SLA)},
    text={SLA},
    plural={SLAs},
    description={ 
        Strumenti contrattuali attraverso i quali si definiscono le metriche che il fornitore del servizio deve garantire. 
    }
}
\newglossaryentry{res}
{
    name={Renewable Energy Sources (RES)},
    first={Renewable Energy Sources (RES)},
    text={RES},
    plural={RESs},
    description={ 
        Fonti di energia rinnovabile, come impianti fotovoltaici, turbine eoliche, etc. 
    }
}
\newglossaryentry{p2p}
{
    name={Peer to Peer (P2P)},
    first={Peer to Peer (P2P)},
    text={P2P},
    plural={P2Ps},
    description={ 
        Modello di connessione che mette in comunicazione diretta i nodi partecipanti o peer. 
    }
}
\newglossaryentry{ew}
{
    name={Energy Web (EW)},
    first={Energy Web (EW)},
    text={EW},
    plural={EWs},
    description={ 
        Progetto che utilizza tecnologie distribuite per realizzare un ecosistema pensato per il settore energetico. 
    }
}
\newglossaryentry{ewdos}
{
    name={Energy Web Decentralized Operating System (EW-DOS)},
    first={Energy Web Decentralized Operating System (EW-DOS)},
    text={EW-DOS},
    plural={EW-DOSs},
    description={ 
        Schema che riassume il progetto di \gls{ew} e ne descrive l'architettura.
    }
}
\newglossaryentry{dapp}
{
    name={Decentralized Application (DApp)},
    first={Decentralized Application (DApp)},
    text={DApp},
    plural={DApps},
    description={ 
        Applicazioni che, invece di appoggiarsi su un backend centralizzato tradizionale, 
        utilizzano un sistema distribuito, come una blockchain. 
    }
}
\newglossaryentry{ewt}
{
    name={Energy Web Token (EWT)},
    first={Energy Web Token (EWT)},
    text={EWT},
    plural={EWTs},
    description={ 
        Token utilizzato su \gls{ewc} 
    }
}
\newglossaryentry{iot}
{
    name={Internet of Things (IoT)},
    first={Internet of Things (IoT)},
    text={IoT},
    plural={IoTs},
    description={ 
        Rete formata da tutti i piccoli dispositivi in grado di interconnettersi e comunicare, 
        generalmente attraverso l'uso di internet. 
    }
}
\newglossaryentry{ens}
{
    name={Ethereum Name Service (ENS)},
    first={Ethereum Name Service (ENS)},
    text={ENS},
    plural={ENSs},
    description={ 
        Servizio simile ad un DNS presente su Ethereum che associa ad un indirizzo esadecimale della blockchain un nome di dominio. 
    }
}
\newglossaryentry{ssi}
{
    name={Self Sovereign Identity (SSI)},
    first={Self Sovereign Identity (SSI)},
    text={SSI},
    plural={SSIs},
    description={ 
        Paradigma che promuove un controllo strettamente personale della propria identità digitale e dei propri dati, 
        senza doverli cedere ad un'autorità centrale. 
    }
}
\newglossaryentry{iam}
{
    name={Identity and Access Management (IAM)},
    first={Identity and Access Management (IAM)},
    text={IAM},
    plural={IAMs},
    description={ 
        Criteri in grado di consentire alle organizzazioni di consentire e controllare gli accessi ad applicazioni, 
        dati e funzionalità solo ad utenti autorizzati. 
    }
}
\newglossaryentry{aggregator}
{
    name={Aggregatore},
    text={aggregatore},
    plural={aggregatori},
    description={
        Un nuovo ruolo che nasce nell'ambito della fornitura di energia e che raggruppa i partecipanti locali (utilizzatori, produttori e \gls{prosumer}). Ha il compito di determinare l'utilizzo dell'energia e può occuparsi della vendita di energia in eccesso prodotta dai \gls{der}.
        L'aggregatore fa da intermediario tra i \gls{prosumer} e i \gls{tso} o i \gls{dso} che si interfacciano con questa realtà l ocale.
    }
}
\newglossaryentry{dso}
{
    name={Distribution System Operator (DSO)},
    first={Distribution System Operator (DSO)},
    text={DSO},
    description={
        Entità che gestiscono (e talvolta possiedono) la rete elettrica a livello regionale o locale.
        Si occupano di principalmente di gestire l'elettricità in ingresso fornita dai \gls{tso} portandola a tensioni adeguate e distribuendola alla rete locale.
    }
}
\newglossaryentry{tso}
{
    name={Transmission System Operator (TSO)},
    first={Transmission System Operator (TSO)},
    text={TSO},
    description={
        Entità responsabile della trasmissione dell'energia elettrica dagli impianti di produzione ai \gls{dso} attraverso la rete elettrica. 
        Si occupano di determinare la quantità di elettricità necessaria per la rete in un dato momento e di gestire le riserve energetiche al fine di evitare pericolosi squilibri nella rete.
    }
}
\newglossaryentry{prosumer}
{
    name={Prosumer},
    text={prosumer},
    description={
        Un utente che svolge entrambi i ruoli di consumatore e produttore di energia. Generalmente possiede uno o più \gls{der}.
    }
}
\newglossaryentry{grid-flexibility}
{
    name={Flessibilità della rete},
    text={flessibilità della rete},
    description={
        L'abilità della rete elettrica di mantenere un equilibrio fra l'energia che genera e quella di cui ha bisogno.
        Sia \glspl{tso} che \glspl{dso} si occupano di fornire flessibilità alla rete.
    }
}
\newglossaryentry{oem}
{
    name={Original Equipment Manufacturer (OEM)},
    first={Original Equipment Manufacturer (OEM)},
    text={OEM},
    plural={OEMs},
    description={
        Azienda che manifattura un prodotto o parti di esso per poi venderlo ad un'altra azienda che lo rivenderà ai propri clienti sotto il suo brand.
    }
}